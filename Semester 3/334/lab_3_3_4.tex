\documentclass[a4paper,12pt]{article}
\usepackage[margin=1in]{geometry}

\usepackage[T2A]{fontenc}			% кодировка
\usepackage[utf8]{inputenc}			% кодировка исходного текста
\usepackage[english,russian]{babel}	% локализация и переносы
\usepackage{graphicx}                % Математика
\usepackage{amsmath,amsfonts,amssymb,amsthm,mathtools} 
\usepackage{mathtext}
\usepackage[T2A]{fontenc}
\usepackage[utf8]{inputenc}

\usepackage{wasysym}

%Заговолок
\author{Бичина Марина 
группа Б04-005 1 курса ФЭФМ}
\title{}
\date{}


\begin{document} % начало документа

\begin{center}
\begin{Large}
{ Марина Б04-005, Лабораторная работа №3.3.4 "Эффект Холла в полупроводниках".}
\end{Large}
\end{center}
\paragraph{Цель работы:} 
\begin{enumerate}
\itemsep0em
\item исследовать зависимость ЭДС Холла от величины магнитного поля при различных значениях тока через образец для определения константы Холла 
\item определить знак носителей заряда и проводимость материала образца
\paragraph{Оборудование:}
\begin{enumerate}
\itemsep0em
\item электромагнит с источником питания
\item амперметр
\item миллиамперметр
\item реостат
\item милливеберметр
\item цифровой вольтметр
\item источник питания (1.5 В)
\item образцы легированного германия
\end{enumerate}
\paragraph{Формулы, необходимые для расчетов:}
\paragraph{}

\paragraph{Описание установки:}
\paragraph{}


\paragraph{Ход работы:}
\begin{enumerate}
\item Откалибруем электромагнит: для этого установим связь между индукцией магнитного поля в зазоре электромагнита и током через обмотку магнита.
\begin{table} [h!]
\begin{center}
\begin{tabular}{|c||c|c|c|c|c|c|c|}
\hline 
$I_m$, А & 0.2 & 0.4 & 0.6 & 0.8 & 1.0 & 1.2 & 1.4 \\ 
\hline 
$B$, мТл & 242.7 & 404.5 & 632.3 & 801.0 & 921.1 & 1003.1 & 1028.3 \\ 
\hline 
\end{tabular}
\end{center}
\end{table}
\item Произведем измерения ЭДС Холла. В таблице приведены значения, при которых начальное на $U_{34}$ напряжение при $I_0=0$ уже учтено и вычтено.
\begin{table} [h!]
\begin{center}
\begin{tabular}{|c||c|c|c|c|c|c|c|}
 \hline 
 № & 1 & 2 & 3 & 4 & 5 & 6 & 7 \\ 
 \hline 
 $I_m$, A & 0.2 & 0.4 & 0.6 & 0.8 & 1.0 & 1.2 & 1.4 \\ 
 \hline 
 \hline
 $U_{0}=66$, мкВ ($I_0=0.3$ А) & 51 & 108 & 158 & 205 & 240 & 263 & 281 \\ 
 \hline 
 $U_{0}=90$, мкВ ($I_0=0.4$ А) & 75 & 143 & 214 & 275 & 322 & 356 & 381 \\ 
 \hline 
 $U_{0} = 112$, мкВ ($I_0=0.5$ А) & 86 & 181 & 266 & 347 & 402 & 445 & 475 \\ 
 \hline 
 $U_{0} = 132$, мкВ ($I_0=0.6$ А) & 101 & 219 & 322 & 379 & 485 & 536 & 572 \\ 
 \hline 
 $U_{0} = 153$, мкВ ($I_0=0.7$ А) & 123 & 251 & 373 & 483 & 568 & 625 & 667 \\ 
 \hline 
 $U_{0} = 175$, мкВ ($I_0=0.8$А) & 140 & 293 & 429 & 552 & 648 & 715 & 762 \\ 
 \hline 
 $U_{0}=197$, мкВ ($I_0=0.9$ А) & 157 & 329 & 482 & 632 & 731 & 804 & 858 \\ 
 \hline 
 $U_{0} = 218$, мкВ ($I_0=1.0$ А) & 172 & 356 & 535 & 693 & 808 & 894 & 953 \\ 
 \hline 
 \end{tabular}
 \end{center}
\end{table}  
\end{enumerate}
\paragraph{Выводы:}
\begin{enumerate}
\item
\end{enumerate}
\end{document}